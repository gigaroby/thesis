\chapter{Analysis of alteratives}
\section{Firechat}
Firechat is a proprietary chat software available for iOS and Android devices.
It is developed by OpenGarden \cite{opengarden}, a company specialized in the creation of peer-to-peer technology.
One of the features of Firechat is the ability to communicate with nearby devices in order to exchange messages in the absence of an Internet connections.
\textbf{Note}: we only tested the Android version of the application.

\paragraph{Nearby Communication} In addition to the various chatrooms available on OpenGarden's servers, Firechat has a ``Nearby'' chatroom that is used to communicate directly with other devices.
In order to spread messages into the ``Nearby'' chatroom the application leverages a combination of communication methods including: UDP broadcast, Bluetooth RFCOMM, Bluetooth Low Energy (BTLE) \cite{btle} and WiFi Direct \cite{wifi-direct}.
Messages sent in the ``Nearby'' chatroom are received by everyone else.

\subsection{Analysis}
Since the source code for the program was not available, some analysis were performed in order to infer the behavior of the application.

\paragraph{Black box testing}
In the first experiment a copy of the Firechat application was downloaded from the Google Play Store \cite{firechat-google-play-store} and installed on three devices.
Cellular and WiFi connections were turned off while Bluetooth was enabled.
Manual pairing was performed on all three devices (each of them was paired with the other two) and then the application was opened.
After a couple of minutes where the messages each phone sent were not received by the other two, the three devices connected in a master-slave setup.

One of the three devices (``A'') received messages sent from the other two (``B'' and ``C'') and was able to send messages to both.
Messages sent by ``B'' were only received by ``A'' but never by ``C'' and vice-versa.
Any attempt to connect more devices to the chat room proved unsuccessful.
This test proved that while a possible communication channel between ``B'' and ``C'' was available through ``A'' the application was not able to exploit it to relay messages.\\

Given that the previous experiment was not able to reliably connect devices via Bluetooth two of the devices were replaced with python scripts running on PCs with a Bluetooth dongle.
In order to communicate to the Firechat application it was necessary to understand how it uses Bluetooth to send messages.
A service discovery scan on a device with the application opened in revealed that two RFCOMM ports were open, in that case number 4 and number 15.
The connection on port 4 did not yield any result but the connection on port 15 returned the last 5 messages sent by that device on the ``Nearby'' chat.

\begin{figure}[ht!]
  \centering
  \includegraphics[width=0.2\textwidth]{img/diagram1.png} 
  \caption{Topology created by three devices on Firechat. Device A could communicate with both B and C but did not pass messages between them}
\end{figure}

\paragraph{Message format}
The messages are serialized as a json object with a few different keys.
While ``user'', ``name'', ``msg'' and ``firechat'' were self-explanatory ``uuid'', ``t'' and ``st'' were not.
``uuid'' uniquely identifies a message to avoid showing it more than once on the interface.
``t'' and ``st'' are two unix timestamps always different by a few seconds.
We were not able to figure out what they represented but they were not needed to perform the tests. 


\begin{minipage}{\textwidth}
  \begin{lstlisting}[caption=Sample Firechat application message,frame=tlrb]{Name}
  {
    "uuid": "7e2b5e4a-d018-4...",
    "name": "John Doe",
    "user": "jdoe01",
    "msg": "hi jack, you there?",
    "firechat": "Nearby",
    "st": 1432828900,
    "t": 1432828895.314917
  }
\end{lstlisting}
\end{minipage}

\paragraph{Bluetooth protocol test}
For the second experiment one device with Firechat (``A'') and two PCs running a custom scripts \cite{custom-script} (``B'' and ``C'') were used. The two PCs were connected to the phone via RFCOMM. In order to verify that ``A'' was not relaying messages, ``B'' was listening on the RFCOMM socket while ``C'' sent a message. While ``A'' received the message correctly it did not send it to ``C''. To verify that ``C'' was actually connected another message was sent by ``A'' and both ``B'' and ``C'' received correctly.

\begin{figure}[ht!]
  \centering
  \includegraphics[width=0.2\textwidth]{img/diagram2.png} 
  \caption{Topology created by connecting two PCs with a device running Firechat. Used to verify that the application did not pass messages between two disjoint networks}
\end{figure}

\subsection{Conclusion}
Even though the application allows some form of local communication it has several problems that make it unfit for a context where discretion is important \cite{iraq-internet-filter} and an Internet connection is not available.

First and foremost all of the messages sent through the application are not encrypted or verified in any way. This makes it trivially easy for an attacker to intercept and tamper with messages.
Additionally if the recipient receives a message it has no way to verify its origin. This is true for both local (``Nearby'') and remote communication.

Finally, the fact that Firechat is not able to relay messages beyond the first hop limits its actual range from a few hundred meters (in case of a big WiFi network that allows UDP broadcast) to a few meters in the average case with Bluetooth.