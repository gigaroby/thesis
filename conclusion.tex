\chapter{Conclusion}
In this thesis we tried to study whether or not it was possible to build a peer to peer messaging protocol using Bluetooth on Android.

After designing a simple protocol based on flooding, we built a series of tests to verify whether Bluetooth primitives allowed an efficient implementation of such protocol.

While results of our tests showed that the throughput in a point to point connection was sufficient even for the transmission of pictures and videos, the amount of time required to setup new connections was a lot greater than we anticipated, especially when dealing with more than one peer.
The tests also highlighted a few stability issues of the Bluetooth API on Android.
Tests would often fail without a specific reason, even in ideal testing conditions.

Slow connection setups had a great impact on the first protocol designed, since a lot of connections had to be established in order to send a message to each peer.
An alternative protocol that minimizes the creation of new connections could perform better, but it would still be affected by the stability issues of Bluetooth on Android.

In the current situation, the development of a complex messaging application relying only on Bluetooth as its communication layer would be very challenging.
In order to minimize the impact of the unreliability of Bluetooth, a better approach may be to create an hybrid system which uses more than one type of connection to carry data between devices.
Such an approach has been proven effective by both OpenGarden in Firechat and Apple with its MultiPeer framework.

