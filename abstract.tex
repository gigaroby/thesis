\begin{abstract}
\pagenumbering{roman}
The goal of this thesis is to investigate whether or not it is possible to implement a peer to peer messaging protocol on modern smartphones using Bluetooth.
The focus has been on Android smartphones in particular, both because Android is by far the most common mobile operating system and because it has a lower barrier for entry than other mobile operating systems such as Apple's iOS.

Bluetooth is a wireless transmission technology created with the purpose of allowing short range communication between mobile devices.
While it is best known for its use in Headsets and In-Car systems, Bluetooth is versatile enough to be used as a general-purpose communication channel.
Android provides APIs for the RFCOMM layer of the Bluetooth protocol stack, which provides a connection-oriented, acknowledged channel with guarantees similar to those of TCP.
Since the RFCOMM protocol emulates a serial cable, it does not support message broadcasting, and in order to send a message to a set of peers using this protocol, a point to point connection with each of those peers has to be established.

Even tough peer to peer networks are a well known problem in computer science, mobile peer to peer networks are not being studied as much.
The few studies that can be found usually employ either WiFi AdHoc or specific hardware in order to create the network.
There are however, a few commercial applications that are based on the same concept and have gained some notoriety after being used in situations where a connection to the Internet was unavailable, such as the Hong Kong protests.
Firechat is the best known application of such kind.
In order to accomplish direct communication, Firechat uses a combination of communication technology available on the device.
For instance, if the device is connected to a WiFi network and has Bluetooth enabled, Firechat will try to send the message both via Bluetooth and via UDP broadcast.
However, during the tests performed, Firechat was unable to connect with all the other smartphones in order to form a mesh network using Bluetooth.
A more targeted test revealed that Firechat does not infact send messages coming from other devices to its peers.
This severely limits its usefulness in situations where a steady Internet connection is not available.

The first step in creating a system that could send messages over a mesh network was to design a very simple protocol to distribute messages and then verify that it worked with Bluetooth.
As it is often the case, the simplest protocol possible was flooding.
A message sent using this protocol is associated with a unique identifier and sent to every peer in the range of the sender.
Then, when a device receives a message it hasn't seen yet, it sends it to every peer it can reach except for the device it was received from.
While this approach ensures that every peer will receive a message reliably, it is most suitable for use with a communication system that supports message broadcasting.
Since the Bluetooth RFCOMM protocol allows for point-to-point communication only, it was important to understand how much the setup of a large number of Bluetooth connection would affect the performance of the system.
To study the cost of the setup of a connection, a test was designed, involving a set of devices which sent a token to each other for a number of times, taking measurement about connection setup times and transmission times.
This test highlighted that the cost of sending of a small text message to another device was largely dominated by the setup of the Bluetooth connection, on average taking 1 to 2 seconds, but going as high as 8.

In order to understand whether Bluetooth would be able to transmit payloads heavier than simple text messages, such as images or videos, it was important to measure the real throughput that the Bluetooth protocol provides.
The test developed for this purpose involved a device sending another device a payload of about 2Mbytes, while taking measures about the performance of the connection.
This test revealed that the Bluetooth RFCOMM protocol provides on average a throughput between 1 and 1.4 Mbps.

Since the performance of data transmission of Bluetooth on Android has proven to be sufficiently high, while the time required to setup a connection is a clear bottleneck, it is evident that a protocol involving messages flooding would not be ideal.
A more suitable alternative would be a protocol where messages are accumulated before sending them to a target, trying to minimize the number of new connections that have to be configured.
Another alternative would be to configure the topology of the network in a semi-static fashion, configuring new connections only when strictly necessary.

While test results showed that, from a performance standpoint, Bluetooth would be sufficient for the task, they also highlighted many of the problems that Bluetooth has -- at least on Android.
Issues were particularly common during the execution of tests involving more that two devices and frequent setups of new connections, a pattern common in massaging applications.
Other problems were simply due to issues in the Bluetooth drivers on a particular device.
While altering the design of the protocol to reduce the number of new connection created might alleviate the problem, the issues are severe enough to make it unsuitable for the development of complex applications which have their focus on direct communication.
A better approach might be to adopt a similar model to that of Firechat, using more than one communication technology at a time in order to achieve better reliability.

After a brief introduction detailing the technological choices in chapter 2, we provide an introduction on Bluetooth.

In chapter four we analyze Firechat, a peer-to-peer application which has gained popularity after being used to circumvent the internet shutdown during the Honk Kong protests.

In chapter five we present our work on the exploration of capabilities of Bluetooth on Android and report performance measured as well as all the issues encountered.

Chapter six concludes the thesis with a discussion of possible future work.
\end{abstract}